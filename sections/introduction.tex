\chapter{Introduction}

In this project I will develop a multi-purpose framework for recommender systems.

As a result of the Digital Revolution and the growth of the Internet we now live in a world of information -- so vast that it is difficult for the user to filter what they seek without assisting tools. Search engines -- such as Google and Bing -- have been developed to help finding information which are related to a search phrase. However this requires the user to build the right search phrase in advance. This leads to the question if meaningful information can be predictively delivered with and without a search phrase. This problem is what recommender systems try to solve -- yet with different approaches.

Recommender systems are usually very tight to the context and architecture of the primary application. Specifically implemented for particular application, it may require to know and access its database structure as well as fit into same implementation constraints such as programming language or database vendor. This project aims to research and develop a solution to easily integrate recommender systems into existing applications with minimal coupling and knowledge of each others' implementation details. The loose coupling will be achieved by using an event-based information exchange via application programming interfaces (API). Thanks to its plug-in mechanism the so can be extended with new recommender algorithms. The solution will be evaluated by demonstrating it in two different use cases.

\section{Proposal Structure}

This proposal discusses specific challenges of recommender systems and -- based on the background research as well as evaluation of existing solutions -- proposes a potential solution to overcome them. Design requirements for the proposed solution are examined and resultant technology choices are defined. Finally, a project plan to ensure the project's execution is compiled.

\begin{description}
    \item[Section 2] introduces the concepts of machine learning and recommender systems. The adaption of recommender systems in research and industry are discussed. Then, an overview of recommender approaches as well as algorithms are described.
    \item[Section 3] discusses challenges of recommender system and proposes a new approach to overcome specific aforementioned challenges. Finally, existing solutions are evaluated.
    \item[Section 4] presents the architectural design of the solution and defines the technologies to be used.
    \item[Section 5] defines a project plan including methodology, evaluation, schedule and a fallback plan in case of complications.
\end{description}